% !TeX TXS-program:compile = txs:///arara
% arara: lualatex: {shell: yes, synctex: no, interaction: batchmode}
% arara: lualatex: {shell: yes, synctex: no, interaction: batchmode} if found('log', '(undefined references|Please rerun|Rerun to get)')

\documentclass[french,a4paper,11pt]{article}
\def\TPversion{0.1.0}
\def\TPdate{17 Mars 2023}
\usepackage[bold-style=ISO,math-style=french]{unicode-math}
\setmainfont{TeX Gyre Schola}
\setmathfont{TeX Gyre Schola Math}
\usepackage{FenetreCas}
\usepackage{awesomebox}
\usepackage{fontawesome5}
\usepackage{enumitem}
\usepackage{tabularray}
\usepackage{fancyvrb}
\usepackage{fancyhdr}
\fancyhf{}
\renewcommand{\headrulewidth}{0pt}
\lfoot{\sffamily\small [FenetreCas]}
\cfoot{\sffamily\small - \thepage{} -}
\rfoot{\hyperlink{matoc}{\small\faArrowAltCircleUp[regular]}}

%\usepackage{hvlogos}
\usepackage{hologo}
\providecommand\tikzlogo{Ti\textit{k}Z}
\providecommand\TeXLive{\TeX{}Live\xspace}
\providecommand\PSTricks{\textsf{PSTricks}\xspace}
\let\pstricks\PSTricks
\let\TikZ\tikzlogo
\newcommand\TableauDocumentation{%
	\begin{tblr}{width=\linewidth,colspec={X[c]X[c]X[c]X[c]X[c]X[c]},cells={font=\sffamily}}
		{\LARGE \LaTeX} & & & & &\\
		& {\LARGE \hologo{pdfLaTeX}} & & & & \\
		& & {\LARGE \hologo{LuaLaTeX}} & & & \\
		& & & {\LARGE \TikZ} & & \\
		& & & & {\LARGE \TeXLive} & \\
		& & & & & {\LARGE \hologo{MiKTeX}} \\
	\end{tblr}
}

\usepackage{hyperref}
\urlstyle{same}
\hypersetup{pdfborder=0 0 0}
\usepackage[margin=1.5cm]{geometry}
\setlength{\parindent}{0pt}
\definecolor{LightGray}{gray}{0.9}

\usepackage[french]{babel}

\usepackage[most]{tcolorbox}
\usetikzlibrary{calc}
\tcbuselibrary{minted}
\NewTCBListing{PresentationCode}{ O{blue} m }{%
	sharp corners=downhill,enhanced,arc=12pt,skin=bicolor,%
	colback=#1!1!white,colframe=#1!75!black,colbacklower=white,%
	attach boxed title to top right={yshift=-\tcboxedtitleheight},title=Code \LaTeX,%
	boxed title style={%
		colframe=#1!75!black,colback=#1!15!white,%
		,sharp corners=downhill,arc=12pt,%
	},%
	top=\baselineskip,%
	fonttitle=\color{#1!90!black}\itshape\ttfamily\footnotesize,%
	listing engine=minted,minted style=colorful,
	minted language=tex,minted options={tabsize=4,fontsize=\small,autogobble},
	#2
}

\tcbset{vignettes/.style={%
	nobeforeafter,box align=base,boxsep=0pt,enhanced,sharp corners=all,rounded corners=southeast,%
	boxrule=0.75pt,left=7pt,right=1pt,top=0pt,bottom=0.25pt,%
	}
}

\tcbset{vignetteMaJ/.style={%
	fontupper={\vphantom{pf}\footnotesize\ttfamily},
	vignettes,colframe=ForestGreen!50!black,coltitle=white,colback=purple!25,%
	overlay={\begin{tcbclipinterior}%
			\fill[fill=purple!75]($(interior.south west)$) rectangle node[rotate=90]{\tiny \sffamily{\textcolor{Black}{\scalebox{0.85}[0.75]{\textbf{MàJ}}}}} ($(interior.north west)+(5pt,0pt)$);%
	\end{tcbclipinterior}}
	}
}

\newcommand\Cle[1]{{\bfseries\sffamily\textlangle #1\textrangle}}
\newcommand\cmaj[1]{\tcbox[vignetteMaJ]{#1}\xspace}

\begin{document}

\setlength{\aweboxleftmargin}{0.07\linewidth}
\setlength{\aweboxcontentwidth}{0.93\linewidth}
\setlength{\aweboxvskip}{8pt}

\pagestyle{fancy}

\thispagestyle{empty}

\vspace{2cm}

\begin{center}
	\begin{minipage}{0.75\linewidth}
	\begin{tcolorbox}[colframe=yellow,colback=yellow!15]
		\begin{center}
			\begin{tabular}{c}
				{\Huge \texttt{FenetreCas [fr]}}\\
				\\
				{\LARGE Des fenêtres CAS à la manière} \\
				\\
				{\LARGE de Xcas ou Geogebra.} \\
			\end{tabular}
			
			\bigskip
			
			{\small \texttt{Version \TPversion{} -- \TPdate}}
		\end{center}
	\end{tcolorbox}
\end{minipage}
\end{center}

\begin{center}
	\begin{tabular}{c}
	\texttt{Cédric Pierquet}\\
	{\ttfamily c pierquet -- at -- outlook . fr}\\
	\texttt{\url{https://github.com/cpierquet/FenetreCas}}
\end{tabular}
\end{center}

\vspace{0.25cm}

{$\blacktriangleright$~~Une commande pour afficher une fenêtre CAS à la manière de \textsf{Xcas}.}

\smallskip

{$\blacktriangleright$~~Une commande pour afficher une fenêtre CAS à la manière de \textsf{Geogebra}.}

\smallskip

{$\blacktriangleright$~~Utilisation de \TikZ{} avec calcul automatique des hauteurs de lignes.}

\smallskip

{$\blacktriangleright$~~Personnalisation de certains éléments (couleurs, etc).}

\smallskip

{$\blacktriangleright$~~Saisie libre des commandes et des résultats.}

\vspace{1cm}

\begin{center}
	\begin{tcolorbox}[enhanced,colframe=cyan,colback=cyan!2,center,width=0.95\linewidth,drop fuzzy shadow=lightgray]
	À la manière de GeoGebra :
	
	\smallskip
	
	\begin{CalculFormelGeogebra}[Largeur=10]
		\LigneCalculsGeogebra%
		{\sffamily g(x)=4/(1+e\textasciicircum(-k x))}
		{$\rightarrow$ \: $\mathsf{g(x)=\dfrac{4}{e^{-kx}+1}}$}
	\end{CalculFormelGeogebra}
	
	\medskip
	
	\hfill~À la manière de Xcas :
	
	\smallskip
	
	\hfill
	\begin{CalculFormelXcas}[Entete=true,Largeur=12]
		\LigneCalculsXcas%
			{\sffamily g(x):=4/(1+e\textasciicircum(-k x))}
			{$\mathsf{x \rightarrow \dfrac{4}{e^{-kx}+1}}$}
	\end{CalculFormelXcas}
	\end{tcolorbox}
\end{center}

\vspace{0.5cm}

%\hfill{}\textit{Merci à Denis Bitouzé et à Gilles Le Bourhis pour leurs retours et idées !}

\smallskip

\vfill

\hrule

\medskip

\TableauDocumentation

\medskip

\hrule

\medskip

\newpage

\phantomsection
\hypertarget{matoc}{}

\tableofcontents

\newpage

\part{Introduction}

\section{Le package FenetreCas}

\subsection{Introduction}

\begin{noteblock}
La package \textit{propose} des outils pour afficher des fenêtres de logiciel de Calcul Formel :

\begin{itemize}
	\item à la manière de \textsf{Xcas} ;
	\item à la manière de \textsf{GeoGebra}.
\end{itemize}
\vspace*{-\baselineskip}\leavevmode
\end{noteblock}

\begin{importantblock}
Les environnements créés sont liés à des environnements \TikZ, et les hauteurs des lignes sont calculées automatiquement -- en interne -- par le package.

\smallskip

L'utilisateur pourra cependant paramétrer \textit{plus finement} le rendu s'il le souhaite.
\end{importantblock}

\subsection{Chargement du package, packages utilisés}

\begin{importantblock}
Le package se charge, de manière classique, dans le préambule.

Il n'existe pas d'option pour la package, et \textsf{xcolor} n'est pas chargé avec des options spécifiques.
\end{importantblock}

\begin{PresentationCode}{listing only}
\usepackage{FenetreCas}
\end{PresentationCode}

\begin{noteblock}
\textsf{FenetreCas} charge les packages suivantes :

\begin{itemize}
	\item \texttt{tikz}, \texttt{xstring}, \texttt{xintexpr}, \texttt{simplekv} et \texttt{settobox} ;
	\item les librairies \texttt{\textit{tikz}.calc} et \texttt{\textit{tikz}.positioning}..
\end{itemize}

Il est compatible avec les compilations usuelles en \textsf{latex}, \textsf{pdflatex}, \textsf{lualatex} ou \textsf{xelatex}.
\end{noteblock}

\section{Fonctionnement global}

\begin{importantblock}
Le fonctionnement global est :

\begin{itemize}
	\item de créer l'\textbf{environnement}, avec ses options \textit{globales} ;
	\item de créer les \textbf{lignes}, une par une, avec leurs options \textit{locales}.
\end{itemize}
\vspace*{-\baselineskip}\leavevmode
\end{importantblock}

\vfill

\section{Historique}

\verb|v0.1.0|~:~~~~Version initiale.

\hspace*{1cm}

\pagebreak

\part{Fenêtre à la manière de Geogebra}

\section{Création de l'environnement}

\subsection{Commande}

\begin{cautionblock}
L'environnement dédié à l'affiche d'une fenêtre \textit{à la Geogebra} est \texttt{CalculFormelGeogebra}.

Par défaut, il va donc créer la ligne d'\textit{entête} avec le titre.
\end{cautionblock}

\begin{PresentationCode}{listing only}
\begin{CalculFormelGeogebra}[clés et options]<options tikz>
\end{CalculFormelGeogebra}
\end{PresentationCode}

\begin{PresentationCode}{}
\begin{CalculFormelGeogebra}
\end{CalculFormelGeogebra}
\end{PresentationCode}

\subsection{Clés et options}

\begin{tipblock}
Le premier argument, optionnel et entre \texttt{[...]} propose les \Cle{clés} suivantes :

\begin{itemize}
	\item \Cle{Largeur} qui est la largeur de la fenêtre, en cm ; \hfill~défaut : \Cle{10}
	\item \Cle{CouleurEntete} qui est la couleur du fond de l'entête ; \hfill~défaut : \Cle{lightgray!25}
	\item \Cle{LargeurNumero} qui est la largeur de la colonne du numéro ; \hfill~défaut : \Cle{1}
	\item \Cle{CouleurNumero} qui est la couleur du fond du numéro ; \hfill~défaut : \Cle{cyan!5}
	\item \Cle{PoliceEntete} qui paramètre la police du texte de l'entête  ;
	
	\hfill~défaut : \Cle{\textbackslash bfseries\textbackslash large\textbackslash sffamily}
	\item \Cle{PoliceNumero} qui paramètre la police du numéro  ; \hfill~défaut : \Cle{\textbackslash Large\textbackslash sffamily}
	\item \Cle{Titre} qui permet de personnaliser le label de l'entête ;
	
	\hfill~défaut : \Cle{\$\textbackslash triangleright\$ Calcul formel}
	\item le booléen \Cle{Entete} qui permet d'afficher ou non l'entête. \hfill~défaut : \Cle{true}
\end{itemize}
\vspace*{-\baselineskip}\leavevmode
\end{tipblock}

\begin{tipblock}
Le second argument, optionnel et entre \texttt{<...>} est quant à lui relatif à des arguments à passer à l'environnement \TikZ{} créé, comme par exemple un alignement vertical, etc
\end{tipblock}

\begin{PresentationCode}{}
\begin{CalculFormelGeogebra}
	[CouleurEntete=green!25,PoliceEntete=\LARGE\ttfamily,
	Titre={$\blacktriangleright$ Illustration via GeoGebra},Largeur=13]
\end{CalculFormelGeogebra}
\end{PresentationCode}

\pagebreak

\section{Création des lignes}

\subsection{Commande}

\begin{cautionblock}
La commande dédiée à l'affiche des lignes est \texttt{\textbackslash LigneCalculsGeogebra}.

Les lignes sont construites l'une après l'une, avec un système de nœuds pour délimiter les \og coins \fg.
\end{cautionblock}

\begin{PresentationCode}{listing only}
\begin{CalculFormelGeogebra}[clés et options]<options tikz>
	LigneCalculsGeogebra[options]{commande}{resultat}
\end{CalculFormelGeogebra}
\end{PresentationCode}

\begin{PresentationCode}{}
\begin{CalculFormelGeogebra}
	\LigneCalculsGeogebra{commande1}{résultat1}
	\LigneCalculsGeogebra{commande2}{résultat2}
\end{CalculFormelGeogebra}
\end{PresentationCode}

\subsection{Clés et arguments}

\begin{tipblock}
Le premier argument, optionnel et entre \texttt{[...]} propose les \Cle{clés} suivantes :

\begin{itemize}
	\item le booléen \Cle{HauteurAuto} qui permet un ajustement automatique de la hauteur ;
	
	\hfill~défaut : \Cle{true}
	\item \Cle{TailleCommande} pour la taille de la commande ; \hfill~défaut : \Cle{\textbackslash normalsize}
	\item \Cle{TailleResultat} pour la taille du résultat ; \hfill~défaut : \Cle{\textbackslash large}
	\item \Cle{MargeV} pour spécifier l'espacement vertical entre les calculs et les traits.
	
	\hfill~défaut : \Cle{6pt}
\end{itemize}
\vspace*{-\baselineskip}\leavevmode
\end{tipblock}

\begin{tipblock}
Les arguments obligatoires, et entre \texttt{\{...\}}, correspondent à la commande et au résultat à afficher dans la ligne :

\begin{itemize}
	\item les tailles des caractères sont fixées par les \Cle{clés} précédemment explicitées ;
	\item la saisie est libre au niveau du contenu, de la police et des couleurs.
\end{itemize}
\end{tipblock}

\begin{PresentationCode}{}
\begin{CalculFormelGeogebra}[CouleurEntete=pink!25,CouleurNumero=yellow!25,Largeur=15]
	\LigneCalculsGeogebra%
		{\sffamily g(x)=4/(1+e\textasciicircum(-k x))}
		{$\rightarrow$ \: $\mathsf{g(x)=\dfrac{4}{e^{-kx}+1}}$}
	\LigneCalculsGeogebra
		{f(x)=1+sqrt(x+3)}
		{$\rightarrow$ \: $f(x)=1+\sqrt{x+3}$}
	\LigneCalculsGeogebra
		{\texttt{Dériver[exp(0.1*x)]}}
		{$\rightarrow$ \: \texttt{x $\mapsto$ 0.1*exp(0.1*x)}}
	\LigneCalculsGeogebra[TailleCommande=\LARGE,TailleResultat=\huge]
		{(1/4+1/3)/(1/5+2/7)}
		{$\rightarrow$ \: $\dfrac{\dfrac14+\dfrac13}{\dfrac15+\dfrac27}=
			\fpeval{(1/4+1/3)/(1/5+2/7)}$}
	\LigneCalculsGeogebra[HauteurAuto=false,HauteurLigne=5]
		{(1+i)\textasciicircum{}2}
		{$2\text{i}$}
\end{CalculFormelGeogebra}
\end{PresentationCode}

\pagebreak

\part{Fenêtre à la manière de Xcas}

\section{Création de l'environnement}

\subsection{Commande}

\begin{cautionblock}
L'environnement dédié à l'affiche d'une fenêtre \textit{à la Xcas} est \texttt{CalculFormelXcas}.

Par défaut, il va donc créer la ligne d'\textit{entête} avec les infos classiques.
\end{cautionblock}

\begin{PresentationCode}{listing only}
\begin{CalculFormelXcas}[clés et options]<options tikz>
\end{CalculFormelXcas}
\end{PresentationCode}

\begin{PresentationCode}{}
\begin{CalculFormelXcas}
\end{CalculFormelXcas}
\end{PresentationCode}

\subsection{Clés et options}

\begin{tipblock}
Le premier argument, optionnel et entre \texttt{[...]} propose les \Cle{clés} suivantes :

\begin{itemize}
	\item \Cle{Largeur} qui est la largeur de la fenêtre, en cm ; \hfill~défaut : \Cle{10}
	\item \Cle{EspaceLg} qui est l'espacement vertical entre les lignes ; \hfill~défaut : \Cle{2pt}
	\item \Cle{Couleur} qui est la couleur des tracés ; \hfill~défaut : \Cle{darkgray}
	\item \Cle{PoliceEntete} qui est la taille de la police de l'entête ; \hfill~défaut : \Cle{\textbackslash scriptsize}
	\item le booléen \Cle{Entete} qui permet d'afficher ou non l'entête. \hfill~défaut : \Cle{true}
	\item le booléen \Cle{Menu} qui permet d'afficher ou non le bouton \textit{MENU} dans les lignes ;
	
	\hfill~défaut : \Cle{true}
	\item le booléen \Cle{NoirBlanc} qui permet de passer en niveaux de gris ; \hfill~défaut : \Cle{false}
	\item \Cle{TexteOptions} qui est le texte des \textit{options} à afficher ;
	
	\hfill~défaut : \Cle{Config : exact real RAD 12 xcas}
	\item le booléen \Cle{Sep} qui permet d'afficher le trait de séparation commande/résultat.
	
	\hfill~défaut : \Cle{true}
\end{itemize}
\vspace*{-\baselineskip}\leavevmode
\end{tipblock}

\begin{tipblock}
Le second argument, optionnel et entre \texttt{<...>} est quant à lui relatif à des arguments à passer à l'environnement \TikZ{} créé, comme par exemple un alignement vertical, etc
\end{tipblock}

\begin{PresentationCode}{}
\begin{CalculFormelXcas}[PoliceEntete=\large,Largeur=13,NoirBlanc]
\end{CalculFormelXcas}
\end{PresentationCode}

\pagebreak

\section{Création des lignes}

\subsection{Commande}

\begin{cautionblock}
La commande dédiée à l'affiche des lignes est \texttt{\textbackslash LigneCalculsXcas}.

Les lignes sont construites l'une après l'une, avec un système de nœuds pour délimiter les \og coins \fg.
\end{cautionblock}

\begin{PresentationCode}{listing only}
\begin{CalculFormelXcas}[clés et options]<options tikz>
	\LigneCalculsXcas[options]{commande}{resultat}
\end{CalculFormelXcas}
\end{PresentationCode}

\begin{PresentationCode}{}
\begin{CalculFormelXcas}
	\LigneCalculsXcas{commande1}{résultat1}
	\LigneCalculsXcas{commande2}{résultat2}
\end{CalculFormelXcas}
\end{PresentationCode}

\subsection{Clés et arguments}

\begin{tipblock}
Le premier argument, optionnel et entre \texttt{[...]} propose les \Cle{clés} suivantes :

\begin{itemize}
	\item \Cle{CouleurCmd} pour la couleur de la commande ; \hfill~défaut : \Cle{red}
	\item \Cle{CouleurRes} pour la couleur du résultat ; \hfill~défaut : \Cle{blue}
	\item \Cle{PosRes} pour la position du résultat ; \hfill~défaut : \Cle{centre}
	\item \Cle{TailleCommande} pour la taille de la commande ; \hfill~défaut : \Cle{\textbackslash normalsize}
	\item \Cle{TailleResultat} pour la taille du résultat ; \hfill~défaut : \Cle{\textbackslash large}
	\item \Cle{MargeV} pour spécifier l'espacement vertical entre les calculs et les traits.
	
	\hfill~défaut : \Cle{6pt}
\end{itemize}
\vspace*{-\baselineskip}\leavevmode
\end{tipblock}

\begin{tipblock}
Les arguments obligatoires, et entre \texttt{\{...\}}, correspondent à la commande et au résultat à afficher dans la ligne :

\begin{itemize}
	\item les tailles des caractères sont fixées par les \Cle{clés} précédemment explicitées ;
	\item la saisie est libre au niveau du contenu, de la police et des couleurs.
\end{itemize}
\end{tipblock}

\begin{PresentationCode}{}
Un exemple en ligne :~
\begin{CalculFormelXcas}%
		[Largeur=10,TexteOptions={Config : exact cpxl RAD 12 xcas}]%
		<baseline=(current bounding box.center)>
	\LigneCalculsXcas%
		{\sffamily g(x)==4/(1+e\textasciicircum(-k x))}
		{$\mathsf{x \rightarrow \dfrac{4}{e^{-kx}+1}}$}
	\LigneCalculsXcas[TailleCommande=\Large,TailleResultat=\LARGE]%
		{\sffamily g(x)=4/(1+e\textasciicircum(-k x))}
		{$\mathsf{x \rightarrow \dfrac{4}{e^{-kx}+1}}$}
	\LigneCalculsXcas
		{f(x)=1+sqrt(x+3)}
		{$x \rightarrow 1+\sqrt{x+3}$}
	\LigneCalculsXcas
		{\texttt{Dériver[exp(0.1*x)]}}
		{\texttt{x $\rightarrow$ 0.1*exp(0.1*x)}}
	\LigneCalculsXcas[TailleResultat=\Huge]
		{(1/4+1/3)/(1/5+2/7)}
		{$\rightarrow$ \: $\dfrac{\dfrac14+\dfrac13}{\dfrac15+\dfrac27}$}
	\end{CalculFormelXcas}
\end{PresentationCode}
%
%\begin{tipblock}
%Concernant ces commandes, qui sont à insérer dans un environnement \textit{math} :
%
%\begin{itemize}
%	\item \cmaj{0.1.3} la version \textit{étoilée} force l'écriture du signe \og $-$ \fg{} sur le numérateur ;
%	\item le premier argument, \textit{optionnel} et entre \textsf{[...]} permet de spécifier un formatage du résultat :
%	\begin{itemize}
%		\item \Cle{t} pour l'affichage de la fraction en mode \textsf{tfrac} ;
%		\item \Cle{d} pour l'affichage de la fraction en mode \textsf{dfrac} ;
%		\item \Cle{n} pour l'affichage de la fraction en mode \textsf{nicefrac} ;
%		\item \Cle{dec} pour l'affichage du résultat en mode \texttt{décimal} (sans arrondi !) ;
%		\item \Cle{dec=k} pour l'affichage du résultat en mode \texttt{décimal} arrondi à $10^{-k}$ ;
%	\end{itemize}
%	\item le deuxième argument, \textit{optionnel} et entre \textsf{<...>} correspond aux \Cle{options} à passer à l'environnement \texttt{pNiceMatrix} ;
%	\item les arguments suivants, \textit{obligatoires} et entre \textsf{(...)}, sont quant à eux, les matrices données par leurs coefficients \textsf{a11,a12,... § a21,a22,...} (syntaxe \textit{inspirée} de \texttt{sympy}) ou la matrice et la puissance ;
%	\item le dernier argument, \textit{optionnel} et entre \textsf{[...]} propose l'unique \frquote{clé} \Cle{Aff} pour afficher le calcul avant le résultat.
%\end{itemize}
%\vspace*{-\baselineskip}\leavevmode
%\end{tipblock}
%
%\begin{PresentationCode}{}
%$\ProduitMatrices(-5,6 § 1,4)(2 § 7)[Aff]$ et $\ProduitMatrices(-5,6 § 1,4)(2 § 7)$
%\end{PresentationCode}
%
%\begin{PresentationCode}{}
%$\ProduitMatrices[dec](0.5,0.3,0.2)(0.75,0.1,0.15 § 0.4,0.4,0.2 § 0.6,0.1,0.3)[Aff]$
%\end{PresentationCode}
%
%\begin{PresentationCode}{}
%$\ProduitMatrices(1,1,1,5 § 2,1,5,6 § 0,5,-6,0 § 1,-5,4,2)(1 § 2 § 3 § 4)[Aff]$
%\end{PresentationCode}
%
%\begin{PresentationCode}{}
%$\ProduitMatrices%
%	(1,1,1,5 § 2,1,5,6 § 0,5,-6,0 § 1,-5,4,2)%
%	(1,5,4,0 § 2,-1,-1,5 § 3,0,1,2, § 4,6,9,10)
%	[Aff]$
%\end{PresentationCode}
%
%\begin{PresentationCode}{}
%$\CarreMatrice(-5,6 § 1,4)[Aff]$
%\end{PresentationCode}
%
%\begin{PresentationCode}{}
%$\CarreMatrice(-5,6,8 § 1,4,-9 § 1,-1,1)[Aff]$
%\end{PresentationCode}
%
%\begin{PresentationCode}{}
%$\MatricePuissancePY(1,1 § 5,-2)(7)[Aff]$
%\end{PresentationCode}
%
%\begin{PresentationCode}{}
%$\MatricePuissancePY(1,1,-1 § 5,-2,1 § 0,5,2)(3)[Aff]$
%\end{PresentationCode}
%
%\begin{PresentationCode}{}
%$\MatricePuissancePY(1,1,1,1 § 5,-2,1,5 § 0,5,2,-1 § 0,1,1,1)(5)[Aff]$
%\end{PresentationCode}
%
%\pagebreak
%
%\section{Calcul de déterminant}
%
%\subsection{Introduction}
%
%\begin{cautionblock}
%Une commande est disponible pour calculer le déterminant d'une matrice :
%
%\begin{itemize}
%	\item \textbf{2x2} ou \textbf{3x3} ou \textbf{4x4}.
%\end{itemize}
%\vspace*{-\baselineskip}\leavevmode
%\end{cautionblock}
%
%\begin{PresentationCode}{listing only}
%%version classique
%\DetMatrice(*)[option de formatage](matrice)
%
%%version python
%\DetMatricePY(*)[option de formatage](matrice)
%\end{PresentationCode}
%
%\subsection{Utilisation}
%
%\begin{tipblock}
%Concernant cette commande, qui est à insérer dans un environnement \textit{math} :
%
%\begin{itemize}
%	\item \cmaj{0.1.3} la version \textit{étoilée} force l'écriture du signe \og $-$ \fg{} sur le numérateur ;
%	\item le premier argument, \textit{optionnel} et entre \textsf{[...]} permet de spécifier un formatage du résultat :
%	\begin{itemize}
%		\item \Cle{t} pour l'affichage de la fraction en mode \textsf{tfrac} ;
%		\item \Cle{d} pour l'affichage de la fraction en mode \textsf{dfrac} ;
%		\item \Cle{n} pour l'affichage de la fraction en mode \textsf{nicefrac} ;
%		\item \Cle{dec} pour l'affichage du résultat en mode \texttt{décimal} (sans arrondi !) ;
%		\item \Cle{dec=k} pour l'affichage du résultat en mode \texttt{décimal} arrondi à $10^{-k}$ ;
%	\end{itemize}
%	\item le second argument, \textit{obligatoire} et entre \textsf{(...)}, est quant à lui, la matrice donnée par ses coefficients \textsf{a11,a12,... § a21,a22,...} (syntaxe \textit{inspirée} de \texttt{sympy}).
%\end{itemize}
%\vspace*{-\baselineskip}\leavevmode
%\end{tipblock}
%
%\begin{PresentationCode}{}
%%version classique
%Le dét. de $A=\AffMatrice(1,2 § 3,4)$ est
%$\det(A)=\DetMatrice(1,2 § 3,4)$.
%\end{PresentationCode}
%
%\begin{PresentationCode}{}
%%version classique
%Le dét. de $A=\AffMatrice[dec](-1,0.5 § 1/2,4)$ est
%$\det(A)=\DetMatrice[dec](-1,0.5 § 1/2,4)$.
%\end{PresentationCode}
%
%\begin{PresentationCode}{}
%%version classique
%Le dét. de $A=\AffMatrice[t](-1,1/3,4 § -1/3,4,-1 § -1,0,0)$ est
%$\det(A) \approx \DetMatrice[dec=3](-1,1/3,4 § -1/3,4,-1 § -1,0,0)$.
%\end{PresentationCode}
%
%\begin{PresentationCode}{}
%Le dét. de $A=\begin{pNiceMatrix} 1&2&3&4\\5&6&7&0\\1&1&1&1\\2&-3&-5&-6 \end{pNiceMatrix}$
%est $\det(A)=\DetMatrice(1,2,3,4 § 5,6,7,0 § 1,1,1,1 § 2,-3,-5,-6)$.
%\end{PresentationCode}
%
%\begin{PresentationCode}{}
%%version python
%Le dé. de $A=\AffMatrice(1,2 § 3,4)$ est
%$\det(A)=\DetMatricePY(1,2 § 3,4)$.
%\end{PresentationCode}
%
%\begin{PresentationCode}{}
%Le dét. de $A=\AffMatrice[dec](-1,0.5 § 1/2,4)$ est
%$\det(A)=\DetMatricePY[d](-1,0.5 § 1/2,4)$.
%\end{PresentationCode}
%
%\begin{PresentationCode}{}
%%version python
%Le dét. de $A=\AffMatrice(-1,1/3,4 § 1/3,4,-1 § -1,0,0)$ est
%$\det(A) \approx \DetMatricePY[dec=3](-1,1/3,4 § 1/3,4,-1 § -1,0,0)$.
%\end{PresentationCode}
%
%\begin{PresentationCode}{}
%%version python
%Le dét. de $A=\AffMatrice(1,2,3,4 § 5,6,7,0 § 1,1,1,1 § 2,-3,-5,-6)$
%est $\det(A)=\DetMatricePY(1,2,3,4 § 5,6,7,0 § 1,1,1,1 § 2,-3,-5,-6)$.
%\end{PresentationCode}
%
%\pagebreak
%
%\section{Inverse d'une matrice}
%
%\subsection{Introduction}
%
%\begin{cautionblock}
%Une commande (matricielle) disponible est pour calculer l'éventuelle inverse d'une matrice :
%
%\begin{itemize}
%	\item \textbf{2x2} ou \textbf{3x3} ou \textbf{4x4} (\cmaj{0.1.5}) pour le package \textit{classique} ;
%	\item \textbf{2x2} ou \textbf{3x3} ou \textbf{4x4} également pour la version \textsf{python}.
%\end{itemize}
%\vspace*{-\baselineskip}\leavevmode
%\end{cautionblock}
%
%\begin{PresentationCode}{listing only}
%%version classique
%\MatriceInverse(*)[option de formatage]<options nicematrix>(matrice)[Clé]
%
%%version python
%\MatriceInversePY(*)[option de formatage]<options nicematrix>(matrice)[Clé]
%\end{PresentationCode}
%
%\subsection{Utilisation}
%
%\begin{tipblock}
%Concernant cette commande, qui est à insérer dans un environnement \textit{math} :
%
%\begin{itemize}
%	\item \cmaj{0.1.3} la version \textit{étoilée} force l'écriture du signe \og $-$ \fg{} sur le numérateur ;
%	\item le premier argument, \textit{optionnel} et entre \textsf{[...]} permet de spécifier un formatage du résultat :
%	\begin{itemize}
%		\item \Cle{t} pour l'affichage de la fraction en mode \textsf{tfrac} ;
%		\item \Cle{d} pour l'affichage de la fraction en mode \textsf{dfrac} ;
%		\item \Cle{n} pour l'affichage de la fraction en mode \textsf{nicefrac} ;
%		\item \Cle{dec} pour l'affichage du résultat en mode \texttt{décimal} (sans arrondi !) ;
%		\item \Cle{dec=k} pour l'affichage du résultat en mode \texttt{décimal} arrondi à $10^{-k}$ ;
%	\end{itemize}
%	\item le deuxième argument, \textit{optionnel} et entre \textsf{<...>} correspond aux \Cle{options} à passer à l'environnement \texttt{pNiceMatrix} ;
%	\item le troisième argument, \textit{obligatoire} et entre \textsf{(...)}, est quant à lui, la matrice donnée par ses coefficients \textsf{a11,a12,... § a21,a22,...} (syntaxe \textit{inspirée} de \texttt{sympy}) ;
%	\item le dernier argument, \textit{optionnel} et entre \textsf{[...]} propose l'unique \frquote{clé} \Cle{Aff} pour afficher le calcul avant le résultat.
%\end{itemize}
%À noter que si la matrice n'est pas inversible, le texte \texttt{Matrice non inversible} est affiché.
%\end{tipblock}
%
%\begin{PresentationCode}{}
%%version classique
%L'inverse de $A=\AffMatrice(1,2 § 3,4)$ est
%$A^{-1}=\MatriceInverse<cell-space-limits=2pt>(1,2 § 3,4)$.
%\end{PresentationCode}
%
%\begin{PresentationCode}{}
%%version classique
%L'inverse de $A=\AffMatrice(1,2,3 § 4,5,6 § 7,8,8)$ est
%$A^{-1}=\MatriceInverse[n]<cell-space-limits=2pt>(1,2,3 § 4,5,6 § 7,8,8)[Aff]$.
%\end{PresentationCode}
%
%\begin{PresentationCode}{}
%%version python
%L'inverse de $A=\AffMatrice(1,2 § 3,4)$ est
%$A^{-1}=\MatriceInversePY[d]<cell-space-limits=2pt>(1,2 § 3,4)[Aff]$.
%\end{PresentationCode}
%
%\begin{PresentationCode}{}
%%version normale
%L'inv. de $A=\AffMatrice(1,2,3,4 § 5,6,7,0 § 1,1,1,1 § -2,-3,-5,-6)$ est
%$A^{-1}=
%\MatriceInverse[n]<cell-space-limits=2pt>(1,2,3,4 § 5,6,7,0 § 1,1,1,1 § -2,-3,-5,-6)$.
%\end{PresentationCode}
%
%\begin{PresentationCode}{}
%%version python
%L'inv. de $A=\AffMatrice(1,2,3,4 § 5,6,7,0 § 1,1,1,1 § -2,-3,-5,-6)$ est
%$A^{-1}=
%\MatriceInversePY[n]<cell-space-limits=2pt>(1,2,3,4 § 5,6,7,0 § 1,1,1,1 § -2,-3,-5,-6)$.
%\end{PresentationCode}
%
%\pagebreak
%
%\section{États avec un graphe probabiliste}
%
%\subsection{Introduction}
%
%\begin{cautionblock}
%\cmaj{0.1.4} Il existe des commandes pour travailler sur un graphe probabiliste (avec le package en version \textsf{python}) :
%
%\begin{itemize}
%	\item afficher un état probabiliste (\textbf{1x2} ou \textbf{1x3} ou \textbf{1x4}, version normale ou version \textsf{python}) ;
%	\item déterminer un état probabiliste à une certaine étape, uniquement en version \textsf{python}.
%\end{itemize}
%\vspace*{-\baselineskip}\leavevmode
%\end{cautionblock}
%
%\begin{PresentationCode}{listing only}
%%version classique ou python
%\AffEtatProb[opt de formatage]<opts nicematrix>(matrice ligne)
%\EtatProbPY[opt de formatage]<opts nicematrix>(état init)(mat de trans)(étape)
%\end{PresentationCode}
%
%\subsection{Utilisation}
%
%\begin{tipblock}
%Concernant la commande d'affichage d'un état, qui est à insérer dans un environnement \textit{math} :
%
%\begin{itemize}
%	\item le premier argument, \textit{optionnel} et entre \textsf{[...]} permet de spécifier un formatage du résultat :
%	\begin{itemize}
%		\item \Cle{t} pour l'affichage de la fraction en mode \textsf{tfrac} ;
%		\item \Cle{d} pour l'affichage de la fraction en mode \textsf{dfrac} ;
%		\item \Cle{n} pour l'affichage de la fraction en mode \textsf{nicefrac} ;
%		\item \Cle{dec} pour l'affichage du résultat en mode \texttt{décimal} (sans arrondi !) ;
%		\item \Cle{dec=k} pour l'affichage du résultat en mode \texttt{décimal} arrondi à $10^{-k}$ ;
%	\end{itemize}
%	\item le deuxième argument, \textit{optionnel} et entre \textsf{<...>} correspond aux \Cle{options} à passer à l'environnement \texttt{pNiceMatrix} ;
%	\item le troisième argument, \textit{obligatoire} et entre \textsf{(...)}, est quant à lui, la matrice donnée par ses coefficients \textsf{a11,a12,...} (syntaxe \textit{inspirée} de \texttt{sympy}).
%\end{itemize}
%\vspace*{-\baselineskip}\leavevmode
%\end{tipblock}
%
%\begin{tipblock}
%Concernant la commande d'affichage d'un état à une étape donnée, qui est à insérer dans un environnement \textit{math} :
%
%\begin{itemize}
%	\item le premier argument, \textit{optionnel} et entre \textsf{[...]} permet de spécifier un formatage du résultat \textsf{[dec] par défaut} :
%	\begin{itemize}
%		\item \Cle{t} pour l'affichage de la fraction en mode \textsf{tfrac} ;
%		\item \Cle{d} pour l'affichage de la fraction en mode \textsf{dfrac} ;
%		\item \Cle{n} pour l'affichage de la fraction en mode \textsf{nicefrac} ;
%		\item \Cle{dec} pour l'affichage du résultat en mode \texttt{décimal} (sans arrondi !) ;
%		\item \Cle{dec=k} pour l'affichage du résultat en mode \texttt{décimal} arrondi à $10^{-k}$ ;
%	\end{itemize}
%	\item le deuxième argument, \textit{optionnel} et entre \textsf{<...>} correspond aux \Cle{options} à passer à l'environnement \texttt{pNiceMatrix} ;
%	\item le troisième argument, \textit{obligatoire} et entre \textsf{(...)}, est quant à lui, la matrice donnée par ses coefficients \textsf{a11,a12,a13} (syntaxe \textit{inspirée} de \texttt{sympy}) ;
%	\item le quatrième argument, \textit{obligatoire} et entre \textsf{(...)}, est quant à lui, la matrice de transition donnée par ses coefficients \textsf{a11,a12,... § a21,a22,...} (syntaxe \textit{inspirée} de \texttt{sympy}) ;
%	\item le cinquième argument, \textit{obligatoire} et entre \textsf{(...)}, est quant à lui, le numéro de l'étape voulue
%\end{itemize}
%\vspace*{-\baselineskip}\leavevmode
%\end{tipblock}
%
%\begin{PresentationCode}{}
%État initial : $P_0 = \AffEtatProb[t](1/3,2/3)$.
%
%Matrice de transition :
%$M=\AffMatrice[dec](0.75,0.25 § 0.9,0.1)$
%
%État à l'instant 5 :
%$P_5 \approx \EtatProbPY[dec=3](1/3,2/3)%
%	(0.75,0.25 § 0.9,0.1)
%	(5)$
%\end{PresentationCode}
%
%\begin{PresentationCode}{}
%État initial : $P_0 = \AffEtatProb[dec](0.33,0.52,0.15)$.
%
%Matrice de transition :
%$M=\AffMatrice[dec]%
%(0.1,0.2,0.7 § 0.25,0.25,0.5 § 0.15,0.75,0.1)$
%
%État à l'instant 7 :
%$P_7 \approx \EtatProbPY[dec=3]
%	(0.33,0.52,0.15)%
%	(0.1,0.2,0.7 § 0.25,0.25,0.5 § 0.15,0.75,0.1)
%	(7)$
%\end{PresentationCode}
%
%\begin{PresentationCode}{}
%État initial : $P_0 = \AffEtatProb[dec](0.33,0.52,0.15,0)$.
%
%Matrice de transition :
%$M=\AffMatrice[dec]%
%(0.1,0.2,0.3,0.4 § 0.25,0.25,0.25,0.25 § 0.15,0.15,0.2,0.5 § 0.3,0.3,0.2,0.2)$
%
%État à l'instant 4 :
%$P_4 \approx \EtatProbPY[dec=3]
%	(0.33,0.52,0.15,0)%
%	(0.1,0.2,0.3,0.4 § 0.25,0.25,0.25,0.25 § 0.15,0.15,0.2,0.5 § 0.3,0.3,0.2,0.2)%
%	(4)$
%\end{PresentationCode}
%
%\pagebreak
%
%\part{Résolution de systèmes}
%
%\section{Résolution d'un système linéaire}
%
%\subsection{Introduction}
%
%\begin{cautionblock}
%Il existe une commande (matricielle) pour déterminer l'éventuelle solution d'un système linéaire qui s'écrit matriciellement $A\times X=B$:
%
%\begin{itemize}
%	\item \textbf{2x2} ou \textbf{3x3} ou \textbf{4x4} (\cmaj{0.1.5}) pour le package \textit{classique} ;
%	\item \textbf{2x2} ou \textbf{3x3} ou \textbf{4x4} également pour le package en version \textsf{python}.
%\end{itemize}
%\vspace*{-\baselineskip}\leavevmode
%\end{cautionblock}
%
%\begin{PresentationCode}{listing only}
%%version classique
%\SolutionSysteme(*)[opt de formatage]<opts nicematrix>(matriceA)(matriceB)[Clé]
%
%%version python
%\SolutionSystemePY(*)[opt de formatage]<opts nicematrix>(matriceA)(matriceB)[Clé]
%\end{PresentationCode}
%
%\subsection{Utilisation}
%
%\begin{tipblock}
%Concernant cette commande, qui est à insérer dans un environnement \textit{math} :
%
%\begin{itemize}
%	\item \cmaj{0.1.3} la version \textit{étoilée} force l'écriture du signe \og $-$ \fg{} sur le numérateur ;
%	\item le premier argument, \textit{optionnel} et entre \textsf{[...]} permet de spécifier un formatage du résultat :
%	\begin{itemize}
%		\item \Cle{t} pour l'affichage de la fraction en mode \textsf{tfrac} ;
%		\item \Cle{d} pour l'affichage de la fraction en mode \textsf{dfrac} ;
%		\item \Cle{n} pour l'affichage de la fraction en mode \textsf{nicefrac} ;
%		\item \Cle{dec} pour l'affichage du résultat en mode \texttt{décimal} (sans arrondi !) ;
%		\item \Cle{dec=k} pour l'affichage du résultat en mode \texttt{décimal} arrondi à $10^{-k}$ ;
%	\end{itemize}
%	\item le deuxième argument, \textit{optionnel} et entre \textsf{<...>} correspond aux \Cle{options} à passer à l'environnement \texttt{pNiceMatrix} ;
%	\item le troisième argument, \textit{obligatoire} et entre \textsf{(...)}, est quant à lui, la matrice $A$ donnée par ses coefficients \textsf{a11,a12,... § a21,a22,...} (syntaxe \textit{inspirée} de \texttt{sympy}) ;
%	\item le quatrième argument, \textit{obligatoire} et entre \textsf{(...)}, est quant à lui, la matrice $B$ donnée par ses coefficients \textsf{b11,b21,...} (syntaxe \textit{inspirée} de \texttt{sympy}) ;
%	\item le dernier argument, \textit{optionnel} et entre \textsf{[...]}, permet -- grâce à la \textit{clé} \Cle{Matrice} -- de présenter le vecteur solution.
%\end{itemize}
%À noter que si la matrice n'est pas inversible, le texte \texttt{Matrice non inversible} est affiché.
%\end{tipblock}
%
%\begin{PresentationCode}{}
%%version classique
%La solution de $\systeme{3x+y-2z=-1,2x-y+z=4,x-y-2z=5}$ est $\mathcal{S}=%
%\left\lbrace \SolutionSysteme[d](3,1,-2 § 2,-1,1 § 1,-1,-2)(-1,4,5) \right\rbrace$.\\
%\end{PresentationCode}
%
%\begin{PresentationCode}{}
%%version python
%La solution de $\systeme{x+y+z=-1,3x+2y-z=6,-x-y+2z=-5}$ est $\mathcal{S}=%
%\left\lbrace \SolutionSystemePY(1,1,1 § 3,2,-1 § -1,-1,2)(-1,6,-5) \right\rbrace$.
%\end{PresentationCode}
%
%\begin{PresentationCode}{}
%%version normal
%La solution de $\systeme[xyzt]{x+2y+3z+4t=-10,5x+6y+7z=0,x+y+z+t=4,-2x-3y-5z-6t=7}$
%est $\mathcal{S}=%
%	\left\lbrace
%		\SolutionSysteme%
%		[dec]<cell-space-limits=2pt>%
%		(1,2,3,4 § 5,6,7,0 § 1,1,1,1 § -2,-3,-5,-6)(-10,0,4,7)%
%\right\rbrace$.
%\end{PresentationCode}
%
%\begin{PresentationCode}{}
%%version python
%La solution de $\systeme[xyzt]{x+2y+3z+4t=-10,5x+6y+7z=0,x+y+z+t=4,-2x-3y-5z-6t=7}$
%est $\mathcal{S}=%
%\left\lbrace
%	\SolutionSystemePY%
%		[dec]<cell-space-limits=2pt>%
%		(1,2,3,4 § 5,6,7,0 § 1,1,1,1 § -2,-3,-5,-6)(-10,0,4,7)%
%\right\rbrace$.
%\end{PresentationCode}
%
%\begin{PresentationCode}{}
%%pas de solution
%La solution de $\systeme{x+2y=-5,4x+8y=1}$ est $\mathcal{S}=%
%\left\lbrace \SolutionSystemePY(1,2 § 4,8)(-5,1) \right\rbrace$.
%\end{PresentationCode}
%
%\pagebreak
%
%\section{Recherche d'un état stable (graphe probabiliste)}
%
%\subsection{Introduction}
%
%\begin{cautionblock}
%\cmaj{0.1.4} Il existe une commande (matricielle) pour déterminer l'éventuel état stable d'un graphe probabiliste :
%
%\begin{itemize}
%	\item \textbf{2x2} pour le package \textit{classique} ;
%	\item \textbf{2x2} ou \textbf{3x3} ou \textbf{4x4} pour le package en version \textsf{python}.
%\end{itemize}
%\vspace*{-\baselineskip}\leavevmode
%\end{cautionblock}
%
%\begin{PresentationCode}{listing only}
%%version classique
%\EtatStable[opt de formatage]<opts nicematrix>(matriceA)
%
%%version python
%\EtatStablePY[opt de formatage]<opts nicematrix>(matriceA)
%\end{PresentationCode}
%
%\subsection{Utilisation}
%
%\begin{tipblock}
%Concernant cette commande, qui est à insérer dans un environnement \textit{math} :
%
%\begin{itemize}
%	\item le premier argument, \textit{optionnel} et entre \textsf{[...]} permet de spécifier un formatage du résultat :
%	\begin{itemize}
%		\item \Cle{t} pour l'affichage de la fraction en mode \textsf{tfrac} ;
%		\item \Cle{d} pour l'affichage de la fraction en mode \textsf{dfrac} ;
%		\item \Cle{n} pour l'affichage de la fraction en mode \textsf{nicefrac} ;
%		\item \Cle{dec} pour l'affichage du résultat en mode \texttt{décimal} (sans arrondi !) ;
%		\item \Cle{dec=k} pour l'affichage du résultat en mode \texttt{décimal} arrondi à $10^{-k}$ ;
%	\end{itemize}
%	\item le deuxième argument, \textit{optionnel} et entre \textsf{<...>} correspond aux \Cle{options} à passer à l'environnement \texttt{pNiceMatrix} ;
%	\item le troisième argument, \textit{obligatoire} et entre \textsf{(...)}, est quant à lui, la matrice donnée par ses coefficients \textsf{a11,a12,... § a21,a22,...} (syntaxe \textit{inspirée} de \texttt{sympy}).
%\end{itemize}
%\vspace*{-\baselineskip}\leavevmode
%\end{tipblock}
%
%\begin{PresentationCode}{}
%%version classique
%L'état stable du gr. prob. de matrice
%$M=\AffMatrice[dec](0.72,0.28 § 0.12,0.88)$
%
%est $\Pi = \EtatStable[d](0.72,0.28 § 0.12,0.88)$
%ou $\Pi = \EtatStable[dec](0.72,0.28 § 0.12,0.88)$.
%\end{PresentationCode}
%
%\begin{PresentationCode}{}
%%version python
%L'état stable du gr. prob. de matrice
%$M=\AffMatrice[dec](0.72,0.28 § 0.12,0.88)$
%
%est $\Pi = \EtatStablePY[d](0.72,0.28 § 0.12,0.88)$
%ou $\Pi = \EtatStablePY[dec](0.72,0.28 § 0.12,0.88)$.
%\end{PresentationCode}
%
%\begin{PresentationCode}{}
%%version python
%L'état stable du gr. prob. de matrice
%$M=\AffMatrice[dec](0.9,0.03,0.07 § 0.30,0.43,0.27 § 0.14,0.07,0.79)$
%
%est $\Pi = \EtatStablePY[d](0.9,0.03,0.07 § 0.30,0.43,0.27 § 0.14,0.07,0.79)$
%ou $\Pi = \EtatStablePY[dec](0.9,0.03,0.07 § 0.30,0.43,0.27 § 0.14,0.07,0.79)$.
%\end{PresentationCode}
%
%\begin{PresentationCode}{}
%%version python
%L'état stable du gr. prob. de matrice
%$M=\AffMatrice[dec]%
%	(0.1,0.2,0.3,0.4 § 0.25,0.25,0.25,0.25 § 0.15,0.15,0.2,0.5 § 0.3,0.3,0.2,0.2)$
%
%est $\Pi \approx
%\EtatStablePY[dec=5]%
%(0.1,0.2,0.3,0.4 § 0.25,0.25,0.25,0.25 § 0.15,0.15,0.2,0.5 § 0.3,0.3,0.2,0.2)$.
%\end{PresentationCode}
%
%
%\pagebreak
%
%\part{Fonctions python utilisées}
%
%\begin{cautionblock}
%Les fonctions utilisées par les packages \textsf{pyluatex} ou \textsf{pythontex} sont données ci-dessous.
%
%Elles sont accessibles en \textit{natif} une fois l'option \textsf{pyluatex} activée, grâce notamment à la macro \texttt{\textbackslash py}.
%\end{cautionblock}
%
%\begin{PresentationCodePython}{listing only}
%#variables symboliques (pour du 4x4 maxi)
%import sympy as sy
%x = sy.Symbol('x')
%y = sy.Symbol('y')
%z = sy.Symbol('z')
%t = sy.Symbol('t')
%\end{PresentationCodePython}
%
%\begin{PresentationCodePython}{listing only}
%#résolution de systèmes
%def resol_systeme_QQ(a,b,c,d,e,f,g,h,i,j,k,l,m,n,o,p,q,r,s,u) :
%	solution=sy.solve([a*x+b*y+c*z+d*t-e,f*x+g*y+h*z+i*t-j,k*x+l*y+m*z+n*t-o,p*x+q*y+r*z+s*t-u],[x,y,z,t])
%	return solution
%
%def resol_systeme_TT(a,b,c,d,e,f,g,h,i,j,k,l) :
%	solution=sy.solve([a*x+b*y+c*z-d,e*x+f*y+g*z-h,i*x+j*y+k*z-l],[x,y,z])
%	return solution
%
%def resol_systeme_DD(a,b,c,d,e,f) :
%	solution=sy.solve([a*x+b*y-c,d*x+e*y-f],[x,y])
%	return solution
%\end{PresentationCodePython}
%
%\begin{PresentationCodePython}{listing only}
%#déterminant d'une matrice
%def det_matrice_QQ(a,b,c,d,e,f,g,h,i,j,k,l,m,n,o,p) :
%	MatTmp = sy.Matrix(([a,b,c,d],[e,f,g,h],[i,j,k,l],[m,n,o,p]))
%	DetMatTmp = MatTmp.det()
%	return DetMatTmp
%
%def det_matrice_TT(a,b,c,d,e,f,g,h,i) :
%	MatTmp = sy.Matrix(([a,b,c],[d,e,f],[g,h,i]))
%	DetMatTmp = MatTmp.det()
%	return DetMatTmp
%
%def det_matrice_DD(a,b,c,d) :
%	MatTmp = sy.Matrix(([a,b],[c,d]))
%	DetMatTmp = MatTmp.det()
%	return DetMatTmp
%\end{PresentationCodePython}
%
%\begin{PresentationCodePython}{listing only}
%#inverse d'une martrice
%def inverse_matrice_QQ(a,b,c,d,e,f,g,h,i,j,k,l,m,n,o,p) :
%	MatTmp = sy.Matrix(([a,b,c,d],[e,f,g,h],[i,j,k,l],[m,n,o,p]))
%	DetMatTmp = MatTmp.inv()
%	return DetMatTmp
%
%def inverse_matrice_DD(a,b,c,d) :
%	MatTmp = sy.Matrix(([a,b],[c,d]))
%	InvMatTmp = MatTmp.inv()
%	return InvMatTmp
%
%def inverse_matrice_TT(a,b,c,d,e,f,g,h,i) :
%	MatTmp = sy.Matrix(([a,b,c],[d,e,f],[g,h,i]))
%	InvMatTmp = MatTmp.inv()
%	return InvMatTmp
%\end{PresentationCodePython}
%
%\begin{PresentationCodePython}{listing only}
%#puissance d'une matrice
%def puissance_matrice_QQ(a,b,c,d,e,f,g,h,i,j,k,l,m,n,o,p,puiss) :
%	MatTmp = sy.Matrix(([a,b,c,d],[e,f,g,h],[i,j,k,l],[m,n,o,p]))
%	PuissMatTmp = MatTmp**puiss
%	return PuissMatTmp
%
%def puissance_matrice_TT(a,b,c,d,e,f,g,h,i,puiss) :
%	MatTmp = sy.Matrix(([a,b,c],[d,e,f],[g,h,i]))
%	PuissMatTmp = MatTmp**puiss
%	return PuissMatTmp
%
%def puissance_matrice_DD(a,b,c,d,puiss) :
%	MatTmp = sy.Matrix(([a,b],[c,d]))
%	PuissMatTmp = MatTmp**puiss
%	return PuissMatTmp
%\end{PresentationCodePython}
%
%\begin{PresentationCodePython}{listing only}
%def etat_prob_QQ(AA,BB,CC,DD,a,b,c,d,e,f,g,h,i,j,k,l,m,n,o,p,puiss) :
%	MatTmpInit = sy.Matrix(([AA,BB,CC,DD])).T
%	MatTmpTrans = sy.Matrix(([a,b,c,d],[e,f,g,h],[i,j,k,l],[m,n,o,p]))
%	EtatProbRes = MatTmpInit * MatTmpTrans**puiss
%	return EtatProbRes
%
%def etat_prob_TT(AA,BB,CC,a,b,c,d,e,f,g,h,i,puiss) :
%	MatTmpInit = sy.Matrix(([AA,BB,CC])).T
%	MatTmpTrans = sy.Matrix(([a,b,c],[d,e,f],[g,h,i]))
%	EtatProbRes = MatTmpInit * MatTmpTrans**puiss
%	return EtatProbRes
%
%def etat_prob_DD(AA,BB,a,b,c,d,puiss) :
%	MatTmpInit = sy.Matrix(([AA,BB])).T
%	MatTmpTrans = sy.Matrix(([a,b],[c,d]))
%	EtatProbRes = MatTmpInit * MatTmpTrans**puiss
%	return EtatProbRes
%\end{PresentationCodePython}
%
%\begin{PresentationCodePython}{listing only}
%def resol_etat_stable_TT(a,b,c,d,e,f,g,h,i) :
%	solution=sy.solve([(a-1)*x+d*y+g*z,b*x+(e-1)*y+h*z,c*x+f*y+(i-1)*z,x+y+z-1],[x,y,z])
%	return solution
%
%def resol_etat_stable_QQ(a,b,c,d,e,f,g,h,i,j,k,l,m,n,o,p) :
%	solution=sy.solve([(a-1)*x+e*y+i*z+m*t,b*x+(f-1)*y+j*z+n*t,
%	c*x+g*y+(k-1)*z+o*t,d*x+h*y+l*z+(p-1)*t,x+y+z+t-1],[x,y,z,t])
%	return solution
%\end{PresentationCodePython}

\end{document}